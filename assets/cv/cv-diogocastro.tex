%% start of file `template.tex'.
%% Copyright 2006-2015 Xavier Danaux (xdanaux@gmail.com).
%
% This work may be distributed and/or modified under the
% conditions of the LaTeX Project Public License version 1.3c,
% available at http://www.latex-project.org/lppl/.

% Install instructions:
% - install texlive using these instructions: https://www.tug.org/texlive/quickinstall.html
%   note: don't use `apt install` because it may not have the latest version
%
% perl install-tl
% - Add `/usr/local/texlive/2021/bin/x86_64-linux` to the PATH
% sudo chown -R dc  /usr/local/texlive/2021/tlpkg/
% tlmgr init-usertree
% tlmgr install etoolbox --verify-repo=none
% pdflatex cv-diogocastro.tex
% - Do a complete reboot (or kill vscode) to allow vscode to pick up
%   the new `PATH`
%
% May need these perl packages to make `latexindent` work:
% sudo cpan YAML::Tiny
% sudo cpan File::HomeDir
% sudo cpan Unicode::GCString


\documentclass[12pt,a4paper,sans]{moderncv}        % possible options include font size ('10pt', '11pt' and '12pt'), paper size ('a4paper', 'letterpaper', 'a5paper', 'legalpaper', 'executivepaper' and 'landscape') and font family ('sans' and 'roman')

% moderncv themes
\moderncvstyle{casual}                             % style options are 'casual' (default), 'classic', 'banking', 'oldstyle' and 'fancy'
\moderncvcolor{blue}                               % color options 'black', 'blue' (default), 'burgundy', 'green', 'grey', 'orange', 'purple' and 'red'
%\renewcommand{\familydefault}{\sfdefault}         % to set the default font; use '\sfdefault' for the default sans serif font, '\rmdefault' for the default roman one, or any tex font name
%\nopagenumbers{}                                  % uncomment to suppress automatic page numbering for CVs longer than one page

% character encoding
\usepackage[utf8]{inputenc}                       % if you are not using xelatex ou lualatex, replace by the encoding you are using

% allow arbitrary font sizes: https://tex.stackexchange.com/a/58088/23314
\usepackage{lmodern}

% adjust the page margins
\usepackage[scale=0.75]{geometry}
\setlength{\hintscolumnwidth}{3.5cm}                % if you want to change the width of the column with the dates
%\setlength{\makecvtitlenamewidth}{10cm}           % for the 'classic' style, if you want to force the width allocated to your name and avoid line breaks. be careful though, the length is normally calculated to avoid any overlap with your personal info; use this at your own typographical risks...

% personal data
\title{Curriculum Vitae}
\name{Diogo}{Castro}
\phone[mobile]{+351 914424559}                     % optional, remove / comment the line if not wanted; the optional "type" of the phone can be "mobile" (default), "fixed" or "fax"
\email{dc@diogocastro.com}                         % optional, remove / comment the line if not wanted
\homepage{diogocastro.com}                         % optional, remove / comment the line if not wanted
\social[github]{dcastro}                           % optional, remove / comment the line if not wanted
\collectionadd[homepage]{socials}{\protect\httplink[diogocastro.com/blog]{diogocastro.com/blog}}


\renewcommand*{\bibliographyitemlabel}{\@biblabel{\arabic{enumiv}}}

\newcommand*{\homepagesocialsymbol}      {{\small\faGlobe}~}

\newcommand*{\para}{\vspace{.7ex}\newline}

\newcommand*{\spacing}{\vspace{.7ex}}

\newcommand*{\code}{\texttt}

\newcommand*{\barelink}[1]{\bluelink{#1}{#1}}
\newcommand*{\bluelink}[2]{\protect\href{#1}{\color{cyan}{#2}}}

\newcommand*{\cvitemspace}{1em}

\renewcommand*{\cvitem}[3][\cvitemspace]{%
  \begin{tabular}{@{}p{\hintscolumnwidth}@{\hspace{\separatorcolumnwidth}}p{\maincolumnwidth}@{}}%
    \raggedleft\hintstyle{#2} &{#3}%
  \end{tabular}%
  \par\addvspace{#1}}

\renewcommand*{\cventry}[7][\cvitemspace]{%
  \cvitem[#1]{#2}{%
    {\bfseries#3}%
    \ifthenelse{\equal{#4}{}}{}{, {\slshape#4}}%
    \ifthenelse{\equal{#5}{}}{}{, #5}%
    \ifthenelse{\equal{#6}{}}{}{, #6}%
    .\strut%
    \ifx&#7&%
    \else{\newline{}\begin{minipage}[t]{\linewidth}#7\end{minipage}}\fi}}

\makeatother

% bibliography with mutiple entries
%\usepackage{multibib}
%\newcites{book,misc}{{Books},{Others}}
%----------------------------------------------------------------------------------
%            content
%----------------------------------------------------------------------------------
\begin{document}

% For some reason, the page counter is off by one
% (it's numering pages as x/4 when there are only 3 pages in total).
%
% This works around the issue by manually setting the total
% page number to 3.
% See: https://stackoverflow.com/a/68292199/857807
\fancyfoot[r]{\parbox[b]{\pagenumberwidth}{\color{color2}\pagenumberfont\strut\thepage/3}}

\setlength{\footskip}{43.49998pt}.

%-----       resume       ---------------------------------------------------------
\makecvtitle

\section{Summary}
\cvitem{}{
  I'm a Software Engineer based in Porto, Portugal.
  \para
  My professional journey began as a C\# developer, but Functional Programming
  soon piqued my interest and led me to learn Scala, Haskell, and PureScript.
  As of late, I've been exploring and using Rust for commercial projects.
  \para
  I love building robust, reliable and maintainable applications.
  \para
  I also like teaching and I'm a big believer in "paying it forward".
  I've learned so much from many inspiring people, so I make it a point to
  share what I've learned with others.
  \para
  I blog about FP and Haskell at \barelink{https://diogocastro.com/blog}.
}

\section{Experience}
\cventry{April 2020-Present}{Team Lead, Software Engineer}{Serokell}{Remote}{}{
  Led the internal-dev team in developing and maintaining several open-source libraries and
  applications written in Haskell, such as
  \bluelink{https://github.com/serokell/xrefcheck/}{xrefcheck}
  and
  \bluelink{https://github.com/serokell/tzbot}{tzbot}.
  Authored \bluelink{https://github.com/serokell/tztime}{tztime}.
  \para
  Led the development of \bluelink{https://certification.haskell.foundation}{Haskell Certification},
  a program for assessing and certifying a candidate's Haskell skills (now operated by the Haskell Foundation).
  \para
  Developed a Rust application using Tauri for controling
  Molecular Rotational Resonance (MRR) spectroscopy instruments,
  perform measurements, and visualize and analyze the measurement data.
  \para
  Helped grow the Haskell and Rust ecosystems for the Tezos blockchain by developing a
  varied set of tools:
  a parser, typechecker and interpreter for the Michelson language, a REPL,
  a test framework for smart contracts, a client library for interacting with the chain, among others.%
  \para
  Led an effort for onboarding new hires and interns.
}

\clearpage

\cventry{Nov 2017-April 2020}{Principal Software Engineer}{SpotX}{Belfast, UK}{}{}
\cventry{}{Senior Software Engineer}{SpotX}{Belfast, UK}{}{
  Developed RESTful web services in Scala, using the \code{cats}/\code{cats-effect} framework and Akka HTTP.
  Used Apache Kafka for publishing of events, and Prometheus/Grafana for monitoring.
  \para
  Worked on a service that aimed to augment and enhance Apache Druid, a timeseries database.
  \para
  Authored a Scala library for calculating the delta of any two values of a given type
  using Shapeless, a library for generic programming.
  \para
  Taught a weekly internal Scala/FP course,
  with the goal of preparing our engineers to be productive in Scala whilst
  building an intuition of how to program with functions and equational
  reasoning.
  \para
  Trained new hires and help them bridge the gap between what they already know
  and the world of FP in Scala.
}

\cventry{May 2013-Nov 2017}{Senior Software Engineer}{NantHealth}{Belfast, UK}{}{}
\cventry{}{Software Engineer}{NantHealth}{Belfast, UK}{}{
  Responsible for designing, unit testing, implementing and deploying a variety of applications, such as:
  \begin{itemize}
    \item RESTful APIs, using ASP.NET Web API, ActiveMQ, Couchbase, and MS SQL Server;
    \item Front-end single-page applications, using AngularJS, TypeScript, LESS;
    \item Internal libraries written in C\#;
    \item Routing of messages between applications using Java and Apache Camel.
  \end{itemize}
  \spacing I've also built some small internal tools to help streamline my coworkers' and
  my day-to-day activities, using Scala, Haskell and PureScript.
  \para
  I helped organise a weekly Brown Bag Session during lunch hour, in which
  people talk about topics that interest them. Sometimes brought an exercise for people to
  solve in a language of their choice, and share their
  solutions at the end.
}
\cventry{Aug 2012-Mar 2013}{Researcher}{Fraunhofer}{Porto, Portugal}{}{
  Developed the navigation module for an Android application, using both turn-by-turn and landmark-based approaches.
  Studied and compared the efficiency of these approaches in navigating older adults with mild dementia.
  Developed complex heuristics to evaluate landmarks data retrieved from OpenStreetMap,
  and used the device's sensors (e.g., gyroscope, accelerometer) to locate and navigate the user.
}

\clearpage

\section{Talks}
\cventry{Jan 2019}{The Haskell Epidemic}{The Crystal Ball BASH, Belfast}{}{}{
  A presentation about some of Haskell's most influential features and how Haskell has shaped the software engineering landscape.
  \para
  Recording: \barelink{https://youtu.be/nnoOF1HeAls}
  \para
  Slides: \barelink{https://talks.diogocastro.com/the-haskell-epidemic/}
}

\section{Education}
\cventry{2007-2013}{Master's in Informatics and Computing Engineering}{Faculty of Engineering of the University of Porto}{Portugal}{}{}

\end{document}

%% end of file `template.tex'.

%% start of file `template.tex'.
%% Copyright 2006-2015 Xavier Danaux (xdanaux@gmail.com).
%
% This work may be distributed and/or modified under the
% conditions of the LaTeX Project Public License version 1.3c,
% available at http://www.latex-project.org/lppl/.


\documentclass[12pt,a4paper,sans]{moderncv}        % possible options include font size ('10pt', '11pt' and '12pt'), paper size ('a4paper', 'letterpaper', 'a5paper', 'legalpaper', 'executivepaper' and 'landscape') and font family ('sans' and 'roman')

% moderncv themes
\moderncvstyle{casual}                             % style options are 'casual' (default), 'classic', 'banking', 'oldstyle' and 'fancy'
\moderncvcolor{blue}                               % color options 'black', 'blue' (default), 'burgundy', 'green', 'grey', 'orange', 'purple' and 'red'
%\renewcommand{\familydefault}{\sfdefault}         % to set the default font; use '\sfdefault' for the default sans serif font, '\rmdefault' for the default roman one, or any tex font name
%\nopagenumbers{}                                  % uncomment to suppress automatic page numbering for CVs longer than one page

% character encoding
\usepackage[utf8]{inputenc}                       % if you are not using xelatex ou lualatex, replace by the encoding you are using

% adjust the page margins
\usepackage[scale=0.75]{geometry}
\setlength{\hintscolumnwidth}{3.5cm}                % if you want to change the width of the column with the dates
%\setlength{\makecvtitlenamewidth}{10cm}           % for the 'classic' style, if you want to force the width allocated to your name and avoid line breaks. be careful though, the length is normally calculated to avoid any overlap with your personal info; use this at your own typographical risks...

% personal data
\title{Curriculum Vitae}
\name{Diogo}{Castro}
\phone[mobile]{+44 7552081034}                     % optional, remove / comment the line if not wanted; the optional "type" of the phone can be "mobile" (default), "fixed" or "fax"
\email{dc@diogocastro.com}                         % optional, remove / comment the line if not wanted
\homepage{diogocastro.com}                         % optional, remove / comment the line if not wanted
\social[github]{dcastro}                           % optional, remove / comment the line if not wanted
\collectionadd[homepage]{socials}{\protect\httplink[diogocastro.com/blog]{diogocastro.com/blog}}


\renewcommand*{\bibliographyitemlabel}{\@biblabel{\arabic{enumiv}}}

\newcommand*{\homepagesocialsymbol}      {{\small\faGlobe}~}

\newcommand*{\para}{\vspace{.7ex}\newline}

\newcommand*{\spacing}{\vspace{.7ex}}

\newcommand*{\project}[3]{
  \cvitem{\textbf{#1}}{
    \barelink{#2}
    \para
    #3
    }
}

\newcommand*{\code}{\texttt}

\newcommand*{\barelink}[1]{\protect\href{#1}{#1}}

\newcommand*{\cvitemspace}{1em}

\renewcommand*{\cvitem}[3][\cvitemspace]{%
  \begin{tabular}{@{}p{\hintscolumnwidth}@{\hspace{\separatorcolumnwidth}}p{\maincolumnwidth}@{}}%
    \raggedleft\hintstyle{#2} &{#3}%
  \end{tabular}%
  \par\addvspace{#1}}

\renewcommand*{\cventry}[7][\cvitemspace]{%
  \cvitem[#1]{#2}{%
    {\bfseries#3}%
    \ifthenelse{\equal{#4}{}}{}{, {\slshape#4}}%
    \ifthenelse{\equal{#5}{}}{}{, #5}%
    \ifthenelse{\equal{#6}{}}{}{, #6}%
    .\strut%
    \ifx&#7&%
    \else{\newline{}\begin{minipage}[t]{\linewidth}#7\end{minipage}}\fi}}

\makeatother

% bibliography with mutiple entries
%\usepackage{multibib}
%\newcites{book,misc}{{Books},{Others}}
%----------------------------------------------------------------------------------
%            content
%----------------------------------------------------------------------------------
\begin{document}

%-----       resume       ---------------------------------------------------------
\makecvtitle

\section{Summary}
\cvitem{}{
  I'm a Software Engineer based in Belfast, United Kingdom.
  \para
  My professional journey began as a C\# developer, but Functional Programming (FP)
  soon piqued my interest and led me to learn Scala, Haskell, PureScript and
  even a bit of Idris.
  \para
  I love building robust, reliable and maintainable applications.
  \para
  I also like teaching and I'm a big believer in "paying it forward".
  I've learned so much from many inspiring people, so I make it a point to
  share what I've learned with others.
  To that end, I'm currently in charge of training new team members in Scala and FP,
  do occasional presentations at work and aim to do more public speaking.
  \para
  I blog about FP and Haskell at \barelink{https://diogocastro.com/blog}.
}

\section{Experience}
\cventry{Nov 2017-Present}{Principal Software Engineer}{SpotX}{Belfast, UK}{}{}
\cventry{}{Senior Software Engineer}{SpotX}{Belfast, UK}{}{
  Developed RESTful web services in Scala, using the \code{cats}/\code{cats-effect} framework and Akka HTTP.
  Used Apache Kafka for publishing of events, and Prometheus/Grafana for monitoring.
  \para
  Worked on a service that aimed to augment Apache Druid, a timeseries database, with features such as
  access control, a safer and simpler query DSL, and automatic conversion of monetary
  metrics to multiple currencies.
  \para
  Authored a Scala library for calculating the delta of any two values of a given type
  using Shapeless, a library for generic programming.
  \para
  Taught a weekly internal Scala/FP course,
  with the goal of preparing our engineers to be productive in Scala whilst
  building an intuition of how to program with functions and equational
  reasoning.
  \para
  I also train new hires and help them bridge the gap between what they already know
  and the world of FP in Scala.
}

\clearpage

\cventry{May 2013-Nov 2017}{Senior Software Engineer}{NantHealth}{Belfast, UK}{}{}
\cventry{}{Software Engineer}{NantHealth}{Belfast, UK}{}{
  Responsible for designing, unit testing, implementing and deploying a variety of applications, such as:
  \begin{itemize}
    \item RESTful APIs, using ASP.NET Web API, ActiveMQ and Couchbase, MS SQL Server;
    \item Front-end single-page applications, using AngularJS, JavaScript / TypeScript, LESS;
    \item Internal libraries (e.g. for standardised logging and messaging) written in C\#;
    \item An internal framework that acts as a concurrent general-purpose service
    host, handling multiple cross-cutting concerns;
    \item Routing of messages between applications using Java and Apache Camel.
  \end{itemize}
  \spacing I've also built some small internal tools to help streamline my coworkers' and
  my day-to-day activities, using Scala, Haskell and PureScript.
  \para
  I helped organise a weekly Brown Bag Session during lunch hour, in which
  people talk about topics that interest them. Sometimes brought a kata (i.e. an
  exercise) for people to solve in a language of their choice, and share their
  solutions at the end.
}
\cventry{Aug 2012-Mar 2013}{Researcher}{Fraunhofer}{Porto, Portugal}{}{
  Developed the navigation module for an Android application, using both turn-by-turn and landmark-based approaches.
  Studied and compared the efficiency of these approaches in navigating older adults with mild dementia.
  Developed complex heuristics to evaluate landmarks data retrieved from OpenStreetMap,
  and used the device's sensors (e.g., gyroscope, accelerometer) to locate and navigate the user.
}

\section{Talks}
\cventry{Jan 2019}{The Haskell Epidemic}{The Crystal Ball BASH, Belfast}{}{}{
  A presentation about some of Haskell's most influential features and how Haskell has shaped the software engineering landscape.
  \para
  Recording: \barelink{https://youtu.be/nnoOF1HeAls}
  \para
  Slides: \barelink{https://talks.diogocastro.com/the-haskell-epidemic/}
}

\clearpage

\section{Open Source}
\project{haskell-flatbuffers}{https://hackage.haskell.org/package/flatbuffers}{
  Haskell implementation of FlatBuffers, a protocol for memory efficient
  serialisation, originally designed by Google.
  It uses TemplateHaskell for generating code from a given schema,
  \code{megaparsec} for parsing schemas and \code{binary} for low-level
  bytestring manipulation.
}
\project{csi-init}{https://github.com/dcastro/csi-init}{
  Csi-init is a simple command line tool written in Haskell, which allows you to
  launch Roslyn's C\# REPL (csi) preloaded with all the assemblies found in one or
  more directories.
}
\project{sequences}{https://github.com/dcastro/sequences}{
  Sequences is a port of Haskell's immutable lazy lists or Scala's \code{Stream[+A]} to C\#.
}
\project{DequeNET}{https://github.com/dcastro/DequeNET}{
  A concurrent lock-free deque (double-ended queue) implementation in C\# -
  push/pop/peek operations in constant time O(1) - and a regular deque
  implemented as a ring buffer.
}
\project{Contributed to AutoFixture}{https://github.com/AutoFixture/AutoFixture}{
  AutoFixture is a .NET library designed to minimise the
  'Arrange' phase of unit tests in order to maximise maintainability.
  It leverages runtime reflection to create complex object graphs
  containing randomised test data.
}
\project{JSend WebApi \& Client}{https://github.com/dcastro/JSendWebApi}{
  JSend.WebApi is an extension of ASP.NET Web API for designing APIs using the
  JSend protocol (\barelink{https://labs.omniti.com/labs/jsend}).
  JSend.Client is a library for consuming JSend APIs.
}
\cvitem{\textbf{Smaller contributions}}{
  \begin{itemize}
    \item mono-traversable: A Haskell library with typeclasses for working with
    both polymorphic and monomorphic containers.
    \item Monocle: An optics library for Scala, inspired by Haskell's \code{lens}.
    \item Circe: A JSON library for Scala.
    \item Refined: A refinement types library for Scala, i.e. enables
    the constraining of types using type-level predicates.
    \item Newts: Scala library with commonly used newtypes and associated typeclass instances.
  \end{itemize}
}

\clearpage

\section{Projects}
\project{2048 AI}{https://github.com/dcastro/twenty48}{
  An AI for the 2048 game using minimax and alpha-beta pruning, as described by John Hughes in the paper "Why Functional Programming Matters".
  The AI was written in Haskell and runs in a Yesod backend.
  The decisions are streamed to the browser via a websockets connection.
  The user's scores are saved in a PostgreSQL database.
  Deployed on AWS using Docker.
  \newline
  Demo: \barelink{https://2048.diogocastro.com/}.
}

\section{Trainings}
\cventry{2016}{JavaScript Training}{Instil Software}{}{}{
  Functional programming in JS, prototype inheritance, Angular.
}
\cventry{2015}{Advanced C\# Course}{Instil Software}{}{}{
  Focused on concurrency, functional programming in C\#, LINQ.
}

\section{Certificates}
\cventry{2014}{Microsoft Certified Professional - Programming in C\#}{Microsoft}{}{}{}
\cventry{2013}{Functional Programming Principles in Scala}{Martin Odersky, École Polytechnique Fédérale de Lausanne}{Coursera}{}{}

\section{Education}
\cventry{2007-2013}{Master's in Informatics and Computing Engineering}{Faculty of Engineering of the University of Porto}{Portugal}{}{}

\section{Skills}
\cvitem{Languages}{
  General purpose: Haskell, Scala, C\#
  \newline{}
  Web: PureScript, JavaScript, TypeScript, CoffeeScript, CSS/Sass/Less
  \newline{}
  SQL: MS SQL Server, MySQL
}
\cvitem{Frameworks}{
  Yesod, cats, Akka HTTP, ScalaCheck/QuickCheck/Hedgehog
}
\cvitem{Tools}{
  Git, Docker, Elastic Search, Apache Druid, Apache Camel, ActiveMQ
}
\cvitem{Other}{
  Experience in an agile setting, using Kanban and Lean principles.
  Property-based testing, test-driven development.
  Typed functional programming.
}


\end{document}

%% end of file `template.tex'.

%% start of file `template.tex'.
%% Copyright 2006-2015 Xavier Danaux (xdanaux@gmail.com).
%
% This work may be distributed and/or modified under the
% conditions of the LaTeX Project Public License version 1.3c,
% available at http://www.latex-project.org/lppl/.


\documentclass[11pt,a4paper,sans]{moderncv}        % possible options include font size ('10pt', '11pt' and '12pt'), paper size ('a4paper', 'letterpaper', 'a5paper', 'legalpaper', 'executivepaper' and 'landscape') and font family ('sans' and 'roman')

% moderncv themes
\moderncvstyle{casual}                             % style options are 'casual' (default), 'classic', 'banking', 'oldstyle' and 'fancy'
\moderncvcolor{blue}                               % color options 'black', 'blue' (default), 'burgundy', 'green', 'grey', 'orange', 'purple' and 'red'
%\renewcommand{\familydefault}{\sfdefault}         % to set the default font; use '\sfdefault' for the default sans serif font, '\rmdefault' for the default roman one, or any tex font name
%\nopagenumbers{}                                  % uncomment to suppress automatic page numbering for CVs longer than one page

% character encoding
\usepackage[utf8]{inputenc}                       % if you are not using xelatex ou lualatex, replace by the encoding you are using

% adjust the page margins
\usepackage[scale=0.75]{geometry}
\setlength{\hintscolumnwidth}{3.5cm}                % if you want to change the width of the column with the dates
%\setlength{\makecvtitlenamewidth}{10cm}           % for the 'classic' style, if you want to force the width allocated to your name and avoid line breaks. be careful though, the length is normally calculated to avoid any overlap with your personal info; use this at your own typographical risks...

% personal data
\name{Diogo}{Castro}
\address{20 Greenvale}{BT17 9LR}{United Kingdom}   % optional, remove / comment the line if not wanted; the "postcode city" and "country" arguments can be omitted or provided empty
\phone[mobile]{07552081034}                        % optional, remove / comment the line if not wanted; the optional "type" of the phone can be "mobile" (default), "fixed" or "fax"
\email{dc@diogocastro.com}                         % optional, remove / comment the line if not wanted
\homepage{diogocastro.com}                         % optional, remove / comment the line if not wanted
\social[github]{dcastro}                           % optional, remove / comment the line if not wanted
\collectionadd[homepage]{socials}{\protect\httplink[diogocastro.com/blog]{diogocastro.com/blog}}


\renewcommand*{\bibliographyitemlabel}{\@biblabel{\arabic{enumiv}}}

\newcommand*{\homepagesocialsymbol}      {{\small\faGlobe}~}   

\newcommand*{\para}{\vspace{.7ex}\newline}

\newcommand*{\spacing}{\vspace{.7ex}}

\newcommand*{\project}[3]{
  \cvitem{#1}{
    \barelink{#2}
    \para
    #3
    }
}
\newcommand*{\code}{\texttt}

\newcommand*{\barelink}[1]{\protect\href{#1}{#1}}

\newcommand*{\cvitemspace}{1em}

\renewcommand*{\cvitem}[3][\cvitemspace]{%
  \begin{tabular}{@{}p{\hintscolumnwidth}@{\hspace{\separatorcolumnwidth}}p{\maincolumnwidth}@{}}%
    \raggedleft\hintstyle{#2} &{#3}%
  \end{tabular}%
  \par\addvspace{#1}}
  
\renewcommand*{\cventry}[7][\cvitemspace]{%
  \cvitem[#1]{#2}{%
    {\bfseries#3}%
    \ifthenelse{\equal{#4}{}}{}{, {\slshape#4}}%
    \ifthenelse{\equal{#5}{}}{}{, #5}%
    \ifthenelse{\equal{#6}{}}{}{, #6}%
    .\strut%
    \ifx&#7&%
    \else{\newline{}\begin{minipage}[t]{\linewidth}#7\end{minipage}}\fi}}

\makeatother

% bibliography with mutiple entries
%\usepackage{multibib}
%\newcites{book,misc}{{Books},{Others}}
%----------------------------------------------------------------------------------
%            content
%----------------------------------------------------------------------------------
\begin{document}

%-----       resume       ---------------------------------------------------------
\makecvtitle

\section{Education}
\cventry{2007-2013}{Master's in Informatics and Computing Engineering}{Faculty of Engineering of the University of Porto}{Portugal}{}{
  Main Topics: Software Engineering, Project Management, Agile Development,
  Mobile Computing, Algorithms and Data Structures, Databases, Web
  Development, Artificial Intelligence, Distributed Systems.
}

\section{Trainings}
\cventry{2015}{Advanced C\# Course}{Instil Software}{}{}{
  Focused on concurrency, functional programming in C\#, LINQ.
}
\cventry{2016}{JavaScript Training}{Instil Software}{}{}{
  Functional programming in JS, prototype inheritance, Angular.
}

\section{Certificates}
\cventry{2013}{Functional Programming Principles in Scala}{Martin Odersky, École Polytechnique Fédérale de Lausanne}{Coursera}{}{}
\cventry{2014}{Microsoft Certified Professional - Programming in C\#}{Microsoft}{}{}{}

\section{Skills}
\cvitem{Languages}{
  General purpose: Scala, Haskell, C\#
  \newline{}
  Web: PureScript, JavaScript, TypeScript, CoffeeScript, CSS/Sass/Less
  \newline{}
  SQL: MS SQL Server, MySQL
}
\cvitem{Frameworks}{
  Yesod, Akka HTTP, ASP.NET, Angular, Apache Camel, WinRT, ScalaCheck/QuickCheck
}
\cvitem{Tools}{
  Git, Docker, Elastic Search, Druid, ActiveMQ, Jenkins, TeamCity, Chef
}
\cvitem{Other}{
  Experience in an agile setting, using Kanban and Lean principles. Property based testing, test driven development.
}


\clearpage

\section{Experience}
\cventry{Aug 2012-Mar 2013}{Researcher}{Fraunhofer}{Porto, Portugal}{}{
Developed the navigation module for an Android application, using both turn-by-turn and landmark-based approaches.
Studied and compared the efficiency of these approaches in navigating older adults with mild dementia.
Developed complex heuristics to evaluate landmarks data retrieved from OpenStreetMap,
and used the device's sensors (e.g., gyroscope, accelerometer) to locate and navigate the user.
}
\cventry{May 2013-Jun 2014}{Software Engineer}{NantHealth}{Belfast, UK}{}{}
\cventry{Jul 2014-Nov 2017}{Senior Software Engineer}{NantHealth}{Belfast, UK}{}{
Responsible for designing, unit testing, implementing and deploying a variety of applications, such as:
\begin{itemize}
  \item RESTful APIs, using ASP.NET Web API, ActiveMQ and Couchbase, Castle Windsor, Bootstrap,
  Coffeescript, LESS, MS SQL Server;
  \item Front-end single page applications, using AngularJS, JavaScript/TypeScript;
  \item Internal libraries (e.g. for standardized logging and messaging) written in C\#;
  \item An internal framework that acts as a concurrent general-purpose service
  host, handling multiple cross-cutting concerns;
  \item Routing of messages between applications using Java and Apache Camel.
\end{itemize}
\spacing I've also built some small internal tools to help streamline my coworkers' and
my day-to-day activities, using Scala, Haskell and PureScript.
\para
I helped organize a weekly Brown Bag Session during lunch hour, in which
people talk about topics that interest them. Sometimes I bring a kata (i.e. an
exercise) for people to solve in a language of their choice, and share their
solutions at the end.
}
\cventry{Nov 2017-Present}{Senior Software Engineer}{SpotX}{Belfast, UK}{}{
Developed RESTful web services in Scala and Akka HTTP, deployed using Docker.%
\para
I teach a weekly internal Scala/Functional Programming course,
with the goal of preparing our engineers to be productive in Scala whilst
building an intuition of how to program with functions and equational
reasoning.
}

\section{Open Source}
\project{Contributed to AutoFixture}{https://github.com/AutoFixture/AutoFixture}{
  AutoFixture is an open source library for .NET designed to minimize the
  'Arrange' phase of your unit tests in order to maximise maintainability. Its
  primary goal is to allow developers to focus on what is being tested rather than
  how to setup the test scenario, by making it easier to create complex object graphs
  containing randomized test data.
}
\project{2048 AI}{https://github.com/dcastro/twenty48}{
  An AI for the 2048 game using minimax and alpha-beta pruning, as described by John Hughes in the paper "Why Functional Programming Matters".
  The AI was written in Haskell and runs in a Yesod backend.
  The decisions are streamed to the browser via a websockets connection.
  Deployed on AWS using Docker. Demo: \barelink{https://2048.diogocastro.com/}.
}
\project{csi-init}{https://github.com/dcastro/csi-init}{
  Csi-init is a simple command line tool written in Haskell, which allows you to
  launch Roslyn's C\# REPL (csi) preloaded with all the assemblies found in one or
  more directories.
}
\project{sequences}{https://github.com/dcastro/sequences}{
  Sequences is a port of Haskell's lists or Scala's \code{Stream[+A]} to C\#.
  A \code{Sequence<T>} is an immutable lazy list whose elements are only evaluated
  when they are needed. It is composed by a head (the first element) and a
  lazily-evaluated tail (the remaining elements).
}
\project{DequeNET}{https://github.com/dcastro/DequeNET}{
  A concurrent lock-free deque (double-ended queue) implementation in C\# -
  push/pop/peek operations in constant time O(1) - and a regular deque
  implemented as a ring buffer.
}
\project{JSend WebApi \& Client}{https://github.com/dcastro/JSendWebApi}{
  JSend.WebApi is an extension of ASP.NET Web API for designing APIs using the
  JSend protocol (\barelink{https://labs.omniti.com/labs/jsend}).
  JSend.Client is a library for consuming JSend APIs.
}
\project{MementoContainer}{https://github.com/dcastro/MementoContainer}{
  An alternative approach to the Memento design pattern. Through reflection,
  the container takes a snapshot of your objects' state, so that you can easily
  rollback when recovering from errors or implementing an "undo" mechanism.
}

\section{Projects}
\project{MetroTasks}{https://www.microsoft.com/en-us/store/apps/metrotasks/9wzdncrdfxdk}{
  Metro Tasks is a productivity application for Windows 8 / Windows RT that
  helps you keep track of your to-do's. The app features synchronization with
  Google Tasks, type-to-search, tile and lock screen updates.
}

\end{document}


%% end of file `template.tex'.

